\chapter{Introduction}

\section{Big Data: What are they?}

Big Data, as the name suggests, are a collection of huge amount of data, that requires a sophisticated infrastructure to be managed and for its valuable information to be extracted.\newline
It is customary to define Big Data as having three fundamental features known as the 3Vs:

\paragraph{Volume}

The quantity of generated and stored must be huge, even in the orders of magnitude of the Brontobytes ($10^{27}$ bytes).
Every bit of data might hide some interesting piece of information. The infrastructure must always retrieve as much data as possible and store it for later use.

\paragraph{Velocity}

In the everchanging environment of our digital world, floating in a widespread network linking every device of our life (Internet of Things), data are flowing at an increasing pace. The infrastructure must be able to keep up with this rhythm, as data ages faster each day.

\paragraph{Variety}

With the rapid increase of published information, known as information explosion, data has begun to diversify more and more, and it is now unrealistic to expect just structured textual information such as classical DataBase tables; the new data flow includes unstructured data such as images, audio, PDFs and raw text files. The infrastructure must be able to extract information from dishomogeneous sources.

\section{A Big Data Infrastructure}\label{Chapter2}

\paragraph{Hardware}

\paragraph{Data \& Cluster Management}

\paragraph{Data Access}

\paragraph{Data Quality}

\paragraph{Data Processing}

\paragraph{Data Ingestion}

\paragraph{Data Visualization}

\paragraph{Management \& Operations}

\paragraph{Security}
